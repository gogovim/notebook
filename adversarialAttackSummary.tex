\documentclass[]{article}
\usepackage{ctex}
\usepackage{graphics}
\usepackage[colorlinks,linkcolor=black,anchorcolor=blue,citecolor=green,CJKbookmarks=True]{hyperref}
\usepackage{amsmath}
\usepackage{CJK}
\usepackage{indentfirst}
\usepackage{amsmath}
\usepackage{mathrsfs}
\usepackage{eufrak}
\usepackage{geometry}



\geometry{a4paper,scale=0.8}

%opening
\title{IJCAI-19阿里巴巴人工智能对抗算法竞赛总结}
\author{wanghao}

\begin{document}

\maketitle

\section{比赛过程}
虽然比赛结果不好,但是还是作下记录。

本次比赛分为三个赛道,防御、无目标攻击、目标攻击。刚看到题目的时候,跟队友讨论了一波,发现防御的跑个分类模型就能得到结果,看起来似乎很简单。于是就决定主要做防御赛道,为比赛gg埋下了伏笔。






\subsection{初赛}

首先在网上找到了防御的三篇论文《Defense Against Adversarial Images using Web-Scale Nearest-Neighbor Search》、《PIXEL DEFEND: LEVERAGING GENERATIVE MODELS TO UNDERSTAND AND DEFEND AGAINST ADVERSARIAL EXAMPLES》、《Retrieval-Augmented Convolutionsl Neural Networks for Improved Robustness against Adversarial Examoles》。觉得第三篇用了图片压缩技术可能在比赛中用不到就只看了前两篇。第一篇论文通过搜索与对抗样本相似的干净样本来防御,pass。第二篇论文通过概率的方法逐像素恢复原图,觉得不靠谱,pass。这几篇文章中提到了FGSM,i-FGSM,CW,DeepFool等攻击方法,初步了解了fgsm。

同时,我们训练了resnet101,inception3,vgg,densenet等基础分类网络,分辨率均为224。除了resnet,其他的效果都不好,resnet101分数达到了14.3263。可能是攻击模型中没有resnet?

看完论坛中大佬的baseline开始补论文,fgsm,对抗训练,集成对抗训练,hgd,随机padding。很多论文中都说fgsm的黑盒迁移性最好,后续就只用了fgsm,连i-fgsm,pgd都没试。。坑。。
然后对resnet101只用fgsm进行了集成对抗训练,效果只提到了14.8641.侥幸进了复赛。赛后向前排大佬请教,发现fgsm的扰动上限设置小了,怪不得毫无效果。

github找到hgd的模型代码,刚好是pytorch的,拿来就直接用了。
\subsection{复赛}

没仔细看hgd的训练过程,知道比赛才发现hgd用的是多个分类模型算的损失。。。

提交之前训好的224分辨率,resnet101模型,效果不好,gg。直接怀疑resize的有效性。重新训练299的分类模型。

提交休战期间训练好的hgd,效果不好,gg。

提交有随机padding但是没有对抗训练的模型,gg。

然后开始找其他去噪模型,看hgd中提到的DAE,论文中说把自编码器和分类网络压在一起形成新的网络,然后同样能产生对抗样本,心里蒙上了一层阴影。然后开始搜一般的盲去噪网络,大半没看懂,放弃。后来找到comdefend,文中说抗干扰能力很强,还不需要对抗样本。发现comdefend中公式含义不清晰,然后误以为网络中有二值化的操作,思考之后发现这个模型的想法真的不错。复现完comdefend模型。训练完后,经过去噪网络后的图片达不到论文中描述的psnr,训练3小时一轮。

提交训好的comdefend模型,效果一般,gg。

无法忍受comdefend模型训练之慢,仿照hgd网络,去掉所有的跨越编解码器的shortcut,用残差块作为基础网络,加入噪声和sigmoid和二值化(不应该加二值化,对论文没理解对),自己设计了个新的去噪网络。新网络的训练效果让我一度对他产生了很大的希望,速度快,准确率高。

提交新的模型,效果一般,gg。

怀疑人生之后,忍无可忍,提交了模型融合。多种去噪网络和多种分类网络,包括集成对抗训练的resnet101和随机padding层。终于达到了最高分,从一堆5、6分的渣渣模型变成了9分。虽然9分也渣。这一天刚好是5.20号,达到了这次比赛的最好成绩。

还剩下10天,天天脑子里就想的就是:

去噪,不用对抗样本!

去噪,不用对抗样本!

去噪,不用对抗样本!

期间尝试在resnet进入全连接层之前做噪声攻击,然后二值化,遇到了后述加噪声模拟攻击的问题。

终于有一天跟师兄讨论,重新理了一遍去噪网络的作用,用数学形式写了一下,推出了一个包含对抗样本的损失函数,联想到comdefend加噪声的方式,忽然间想到可以在编码器后的输出加入噪声模拟攻击即可。兴奋地和队友讨论,增强信心之后,实现了这个思路。期间发现了梯度截断的问题,借鉴了二值化网络的方法,用了hardtanh。

提交最新的模型,gg。

发现在编码器后面加噪声并不能解决问题,前面的编码器网络的参数极有可能在原来的基础上集体变大10倍、100倍,使网络的输出整体变大以达到抗干扰的目的。再度陷入迷茫,决定采用fgsm生成的对抗样本,不使用噪声模拟攻击。

提交新模型,gg。

返回的分数跟训练效果完全不一致,缩小为测试的四倍,开始怀疑对抗样本生成有问题。在l1限制的fgsm上进行修改,只取梯度绝对值前20\%的像素点改变梯度。用此方法攻击提交的模型,模型效果果然很差,修改后继续训练。此时5.28号。

一直在想怎么根据各点的梯度大小动态调整要攻击的像素点,懵逼地发现我应该用l2限制的fgsm。。修改后继续训练。此时5.29号,以为5.31号才结束,心里稳定得一批(fyzz),看着群里大佬熬夜苦战。

5.30号,惊悚发现早上10点结束。

提交防御新模型,返回nvidia-docker error,心态炸裂,发现忘记把压缩位数改为8了,超过10点,gg。9.0309分,排名70。

提交l2的fgsm和降80\%像素点梯度置零的方法,无目标攻击通道提升了几名,gg。44.3563分,排名96。
\section{比赛总结}

\begin{enumerate}
	\item 同时做攻击赛道和防御赛道能较早发现从头持续到尾的错误,单人作战极易在错误的道路上越走越远,申清题意。
	\item 要尝试突破原有认知,不能看多了fgsm迁移性好就不尝试其他的攻击方法,不能看到论文中大部分扰动很小就不尝试大扰动。
	\item 多模型真的很有用。论文的总结性的话只能信一半,不同的数据集,效果不一定一样。
	\item 不确定的或可疑的信息,一定要确认,否则当作不存在。
	\item 比赛时,不同的方法一定要多尝试。继续战斗。
\end{enumerate}

\end{document}

